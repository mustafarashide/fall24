\documentclass[12pt,letterpaper, onecolumn]{exam}
\usepackage{amsmath}
\usepackage{amssymb}
\usepackage{graphicx}
\usepackage{caption}
\graphicspath{ {./images/} }
\usepackage[lmargin=71pt, tmargin=1.2in]{geometry}
\lhead{Multivariable Calculus\\}
\rhead{Midterm I\\}
\chead{\hline} 
\thispagestyle{empty} 
\newcommand*{\setdef}[1]{\left\{#1 \right\}} 

\begin{document}
	\begingroup  
	\centering
	\LARGE Multi-variable Calculus\\
	\LARGE Midterm I\\[0.5em]
	\large \today\\[0.5em]
	\large Fall 2024\par
	\endgroup
	
	\noindent 12.1 Functions of Two Variables\\
	\begin{itemize}
		\item 	Distance between point s $(x,y,z)$ and $a,b,c$ in 3-space
		$$=\sqrt{(x-a)^2+(y-b)^2+(z-c)^2}$$
		
	\end{itemize}

		\noindent 12.2 Graphs \& Surfaces\\
		\begin{itemize}
		\item The graph of a function of two variables, $f$, is the set of all points $(x,y,z)$ such that $z=f(x,y)$. In general the graph of a function of two variables is a surface in 3-space.
		\item For a function $f(x,y)$, the function we get by holding $x$ fixed and letting $y$ vary is called a \textbf{cross-section} of $f$ with $x$ fixed. The graph of the cross-section of $f(x,y)$ with $x=c$ is the curve or cross-section, we get by intersecting the graph of $f$ with the plane $x=c$. We define a cross-section of $f$ with $y$ fixed similarly. 
		\end{itemize}
		\noindent 12.3 Contour Diagrams
		\begin{itemize}
			\item Contour diagrams are used to represent functions of two variables as they are difficult to see function behavior from a surface
			\item Contour lines, or level curves, are obtained from a surface by slicing it with horizontal planes. A contour diagram is a collection of level curves labeled with function values.
		\end{itemize}
\end{document}