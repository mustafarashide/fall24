\documentclass[12pt,letterpaper, onecolumn]{exam}
\usepackage{amsmath,graphicx}
\usepackage{amssymb}
\usepackage{amsthm}
\usepackage{enumitem}
\usepackage{caption}
\usepackage{array}
\usepackage[dvipsnames]{xcolor}
\usepackage[lmargin=71pt, tmargin=1.2in]{geometry}
\lhead{Mustafa Rashid\\}
\rhead{Chapters 5.9,6.1 \& 6.2\\}
\chead{\hline} 
\thispagestyle{empty} 
\newcommand*{\setdef}[1]{\left\{#1 \right\}} 
\newcommand{\doesnotdivide}{\not\hspace{2.5pt}\mid}

\begin{document}
	
	\begingroup  
	\noindent\LARGE Discrete Mathematics\\
	\noindent\LARGE Chapters 5.9,6.1 \& 6.2 Homework\\
	\noindent\large \today\\
	\noindent\large Mustafa Rashid\par
	\noindent\large Fall 2024\par
	\endgroup
	\rule{\textwidth}{0.4pt}
	\pointsdroppedatright
	\printanswers
	\renewcommand{\solutiontitle}{\noindent\textbf{Ans:}\enspace}  
	
	\centering\large Exercise Set 5.9\\
	\begin{questions}
	\setcounter{question}{5}\question Define a set S recursively as follows:
	\begin{enumerate}[label=\Roman*.]
		\item BASE: $a\in S$
		\item RECURSION: If $s\in S$, then,\\
		a. $sa \in S$\hspace*{2.5cm}b. $sb\in S$
		\item RESTRICTION: Nothing is in S other than objects defined in I and II above.
	\end{enumerate}
	Use structual induction to prove that every string in S begins with an $a$.
	\begin{solution}
		\begin{proof}
		Given any string $s\in S$, let the property be the claim that it begins with an $a$.\\
		\textit{\textbf{Show that each object in the BASE case for $S$ satisfies the property:}} The only object in the base case is $a$ which satisfies the property as it begins with $a$.\\
			\textit{\textbf{Show that for each rule in the RECURSION for S, if the rule is applied to an object in S that satisfies the property, then the objects defined by the rule also satisfy the property:}} The recursion for $S$ consists of two rules denoted II(a) and II(b). Suppose $s$ is a string in $S$ that begins with $a$. In the case where the rule II(a) is applied to $s$, the result is the string $sa$, which also begins with $a$. In the case where the rule II(b) is applied to $s$, the result is the string $sb$, which also begins with $a$. Thus when each rule in the RECURSION is applied to a string in $S$ that begins with $a$, the result is also a string that begins with $a$.
		\end{proof}
	\end{solution}
	\setcounter{question}{10}\question Define a set S recursively as follows:
		\begin{enumerate}[label=\Roman*.]
		\item BASE: $0\in S$
		\item RECURSION: If $s\in S$, then,\\
		a. $s+3 \in S$\hspace*{2.5cm}b. $s-3\in S$
		\item RESTRICTION: Nothing is in S other than objects defined in I and II above.
	\end{enumerate}
	Use structual induction to prove that integer in S is divisible by 3.
	\begin{solution}
		\begin{proof}
		Given any integer $s\in S$, let the property be the claim that it is divisibly by 3.\\
		\textit{\textbf{Show that each object in the BASE case for $S$ satisfies the property:}} The only object in the base case is $0$ which satisfies the property because by definition $3\mid 0,  \exists k\in\mathbb{Z},  0=3k$.\\
			\textit{\textbf{Show that for each rule in the RECURSION for S, if the rule is applied to an object in S that satisfies the property, then the objects defined by the rule also satisfy the property:}} The recursion for $S$ consists of two rules denoted II(a) and II(b). Suppose $s$ is an integer in $S$ that is divisible by 3. Then by definition $\exists l \in\mathbb{Z}, s=3l$. In the case where the rule II(a) is applied to $s$, the result is the integer $s+3=3l+3=3(l+1)$, which is also divisible by 3 by definition because $\exists m \in\mathbb{Z}, s+3=3(l+1)$ where $m=l+1$. In the case where the rule II(b) is applied to $s$, the result is the integer $s-3=3l-3=3(l-1)$, which is also divisible by 3 by defnition because $\exists n\in\mathbb{Z}, s-3=3(l-1)$ where $n=l-1$. Thus when each rule in the RECURSION is applied to an integer $s$ that is divisible by 3, the result is also an integer that is divisible by 3.
	\end{proof}
	\end{solution}
	\setcounter{question}{15}\question Give a recursive definition for the set of all strings of 0's and 1's for which all the 0's precede all the 1's.
	\begin{solution}
		Let $S$ be the set of all strings of 0's and 1's for which all the 0's precede all the 1's. The following is the recursive definition of the set $S$:
			\begin{enumerate}[label=\Roman*.]
			\item BASE: $0\in S$
			\item RECURSION: If $s\in S$, then,
			\begin{enumerate}[label=\alph*.]
				\item $s1\in S$
				\item $0s\in S$
			\end{enumerate}
			\item RESTRICTION: Nothing is in S other than objects defined in I and II above.
			\end{enumerate}
	\end{solution}
	\setcounter{question}{17}\question Give a recursive definition for the set of all strings of $a$'s and $b$'s that contain exactly one $a$.
	\begin{solution}
	Let $S$ be the set of all strings of $a$'s and $b$'s that contain exactly one $a$. The following is the recursive definition of the set $S$:
	\begin{enumerate}[label=\Roman*.]
		\item BASE: $a\in S$
		\item RECURSION: If $s\in S$, then,
		\begin{enumerate}[label=\alph*.]
			\item $sb\in S$
			\item $bs\in S$
		\end{enumerate}
		\item RESTRICTION: Nothing is in S other than objects defined in I and II above.
	\end{enumerate}
	\end{solution}
	\setcounter{question}{24}\question Student C tries to define a function $G:\mathbb{Z^{+}\rightarrow Z}$ by the rule
	\begin{equation*}
		G(n)=
		\begin{cases}
			1 & \text{if } n \text{ is 1}}\\
			G\left(\frac{n}{2}\right)&\text{if } n \text{ is even}}\\
			2+G(3n-5)&\text{if } n \text{ is odd and $n>1$}}\\
		\end{cases}
	\end{equation*}
	for all integers $n\geq1$. Student $D$ claims that $G$ is not well defined. Justify  $D$'s claim.
	\begin{solution}
		Suppose $G$ is a function, then by definition of $G$,
		\begin{align*}
			G(1)&=1,\\
			G(2)&=G(1)=1,\\
			G(3)&=2+G(4)=2+G(2)=2+1=3,\\
			G(4)&=G(2)=1,\\
		\text{However,}\\
			G(5)&=2+G(10)=2+G(5).
		\end{align*}
		Subtracting $G(5)$ from both sides gives $0=2$ which is false. Since supposing $G$ is a function leads to a logically false statement, then $G$ cannot be a function and Student D's claim is justified.
	\end{solution} 
	\end{questions}
	
	\centerline{Exercise Set 6.1}
	\begin{questions}
	\setcounter{question}{3}\question Let $A=\{n\in\mathbb{Z}:n=5r \text{ for some integer }r\}$ and $B=\{m\in\mathbb{Z}:m=20s \text{ for some integer } s\}.$
	\begin{parts}
		\part Is $A\subseteq B$? Explain.
		\part Is $B\subseteq A$? Explain.
	\end{parts}
	\begin{solution}
		\begin{parts}
			\part No, $\exists n\in A, n\notin B$. For example $5\in A, 5\notin B$
			\part Yes. $\forall m \in B, m\in A$. Because $m\in B$, then by definition of set $B$, $m=20s, s\in\mathbb{Z}$ or $m=5\cdot4s=5l, l\in\mathbb{Z}$ and therefore $m\in A$ by the definition of set $A$.
		\end{parts}
	\end{solution}
	\setcounter{question}{6} \question  
	Let \begin{align*}
		A&=\{x\in\mathbb{Z}:x=6a+4 \text{ for some integer } a\},\\
		B&=\{y\in\mathbb{Z}:y=18b-2\text{ for some integer } b\},\\
		\text{and}\\
		C&=\{z\in\mathbb{Z}:z=18c+16\text{ for some integer }c\}.
	\end{align*}
	Prove or disprove each of the following statements.
	\begin{parts}
		\part $A\subseteq B$
		\part $B\subseteq A$
		\part $B=C$
	\end{parts}
	\begin{solution}
		\begin{parts}
			\part \begin{proof}
			Suppose $x$ is a particular but arbitrarily chosen element of $A$, then by definition $x=6a+4$ where $a\in\mathbb{Z}$. Suppose $a=0$, and so $x=4$ meaning that $4\in A$. However, $4\notin B$ because there is no integer solution for the equation $18b-2=4$. Therefore $A\nsubseteq B$ because $\exists x \in A, x\notin B$.
					\end{proof}
			\part \begin{proof}
			Suppose $y$ is a particular but arbitrarily chosen element of $B$, then by definition $y=18b-2$ where $b\in\mathbb{Z}$.  Let $a=3b-1\in\mathbb{Z}$ because $b\in\mathbb{Z}$ and the products of integers are integers and so is the difference of integers. Then, $6a+4=6(3b-1)+4=18b-2=x$, and so by definition of $A$, $x$ is an element of $A$.
			\end{proof}
			\part \begin{proof}
			Yes. If $B \subseteq C \land C \subseteq B \rightarrow B=C$\\
			\textbf{1) Proving that $B\subseteq C:}$}\\
			Suppose $y$ is a particular but arbitrarily chosen element of $B$, then by definition $y=18b-2$ where $b\in\mathbb{Z}$. Let $c=b-1\in\mathbb{Z}$ because the difference of integers is also an integer. Then $18c+16=18(b-1)+16=18b-2=y$, and so by definition of $C$, $y$ is an element of $C$.\\
			\textbf{2) Proving that $C\subseteq B:$}\\Suppose $z$ is a particular but arbitrarily chosen element of $C$, then by definition $z=18c+16$ where $c\in\mathbb{Z}$. Let $b=c+1\in\mathbb{Z}$ because the sum of integers is also an integer. Then $18(c+1)-2=18c+16=z$, and so by definition of $B$, $z$ is an element of $B$.\\
			Because $B \subseteq C \land C \subseteq B$ then $B=C$.
			
			\end{proof}
		\end{parts}
	\end{solution}
	\setcounter{question}{11}\question Let the universal set be the set $\mathbb{R}$ of all real numbers and let 
	 \begin{align*}
		A&=\{x\in\mathbb{R}:-3\leq x \leq 0\},\\
		B&=\{x\in\mathbb{R}:-1<x<2\},\\
		\text{and}\\
		C&=\{x\in\mathbb{R}:6<x\leq8\}.
	\end{align*}
	Find each of the following:
	\begin{parts}
		\part $A\cup B$
		\part $A\cap B$
		\part $A^c$
		\part $A\cup C$
		\part $A\cap C$
		\part $B^c$
		\part $A^c \cap B^c$
		\part $A^c \cup B^c$
		\part $(A \cap B)^c$
		\part $(A \cup B)^c$
	\end{parts}
	\begin{solution}
		\begin{parts}
			\part $\{x\in\mathbb{R} : -3\leq x <2\}$
			\part $\{x\in\mathbb{R}:-1<x\leq0\}$
			\part $\{x\in\mathbb{R}:x<-3 \lor x>0\}$
			\part $\{x\in\mathbb{R}:-3\leq x\leq0 \lor 6<x\leq8\}$
			\part $\{\}$
			\part $\{x\in\mathbb{R}:x\leq-1 \lor x\geq2\}$
			\part $\{x\in\mathbb{R}:x<-3 \lor x\geq2\}$
			\part $\{x\in\mathbb{R}:x\leq-1 \lor x>0\}$
			\part $\{x\in\mathbb{R}:x\leq-1 \lor x>0\}$
			\part $\{x\in\mathbb{R}:x<-3 \lor x\geq2\}$
		\end{parts}
	\end{solution}
	\setcounter{question}{16}\question Consider the Venn diagram shown below. For each of (b)-(f), copy the diagram and shade the region corresponding to the indicated set.
	\begin{center}
		\includegraphics[width=0.4\linewidth]{vendiagram}
		\caption{}
		\label{fig:vendiagram}
	\end{center}
	
	\begin{parts}
		\setcounter{partno}{1}\part $B\cup C$ 
		\part $A^c$
		\part $A-(B\cup C)$
		\part $(A\cup B)^c$
		\part $A^c \cap B^c$ 
	\end{parts}
	\begin{solution}\\
	\begin{tabular}{cc}
		b) \includegraphics[width=0.3\linewidth]{17b} & c) \includegraphics[width=0.3\linewidth]{17c} \\
		d) \includegraphics[width=0.3\linewidth]{17d} & e) \includegraphics[width=0.3\linewidth]{17e} \\
		f) \includegraphics[width=0.3\linewidth]{17e} & \\
	\end{tabular}
	\end{solution}
	\setcounter{question}{19}\question Let $B_i=\{x\in\mathbb{R}:0\leq x\leq i\}$ for all integers $i=1,2,3,4.$
	\begin{parts}
		\part $B_1\cup B_2 \cup B_3 \cup B_4=?$
		\part $B_1 \cap B_2 \cap B_3 \cap B_4=?$
		\part Are $B_1,B_2,B_3,$ and $B_4$ mutually disjoint? Explain?
	\end{parts}
	\begin{solution}
		\begin{parts}
				\part $\{x\in\mathbb{R}:0\leq x\leq4\}$
			\part $\{x\in\mathbb{R}:0\leq x\leq1\}$
			\part No, they are not mutually disjoint. There is at least one set $B_i \cap B_j$ where $i\neq j$ that is not empty. For example, $B_1\cap B_4=\{0\leq x \leq 1\}$.
		\end{parts}
	\end{solution}
	\setcounter{question}{29}\question Let $\mathbb{Z}$ be the set of all integers and let
	\begin{align*}
		A_0&=\{n\in\mathbb{Z}:n=4k, \text{ for some integer }k\},\\
		A_1&=\{n\in\mathbb{Z}:n=4k+1, \text{ for some integer }k\},\\
		A_2&=\{n\in\mathbb{Z}:n=4k+2, \text{ for some integer }k\},\\
		A_3&=\{n\in\mathbb{Z}:n=4k+3, \text{ for some integer }k\},
	\end{align*}
	Is $\{A_0,A_1,A_2,A_3\}$ a partition of $\mathbb{Z}$? Explain your answer.
	\end{questions}
	
\end{document}