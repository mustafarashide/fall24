\documentclass[12pt,letterpaper, onecolumn]{exam}
\usepackage{amsmath}
\usepackage{amssymb}
\usepackage{amsthm}
\usepackage{graphicx}
\usepackage{caption}
\usepackage{array}
\usepackage[dvipsnames]{xcolor}
\usepackage[lmargin=71pt, tmargin=1.2in]{geometry}
\lhead{Mustafa Rashid\\}
\rhead{Chapter 5.1\\}
\chead{\hline} 
\thispagestyle{empty} 
\newcommand*{\setdef}[1]{\left\{#1 \right\}} 
\newcommand{\doesnotdivide}{\not\hspace{2.5pt}\mid}

\begin{document}
	
	\begingroup  
	\noindent\LARGE Discrete Mathematics\\
	\noindent\LARGE Chapter 5.1 Homework\\
	\noindent\large \today\\
	\noindent\large Mustafa Rashid\par
	\noindent\large Fall 2024\par
	\endgroup
	\rule{\textwidth}{0.4pt}
	\pointsdroppedatright
	\printanswers
	\renewcommand{\solutiontitle}{\noindent\textbf{Ans:}\enspace}  
	
	\centering\large Exercise Set 5.1\\
	\begin{questions}
		\setcounter{question}{8} \question  Find explicit formulas for the sequence of the form $a_1,a_2,a_3,...$ with the initial terms below
		$$1-\frac{1}{2},\frac{1}{2}-\frac{1}{3},\frac{1}{3}-\frac{1}{4},\frac{1}{4}-\frac{1}{5},\frac{1}{5}-\frac{1}{6},\frac{1}{6}-\frac{1}{7}$$
		
		\begin{solution}
			$$a_k=\frac{1}{k}-\frac{1}{k+1} \textrm{  for all integers } k\geq0$$
		\end{solution}
		\setcounter{question}{28} \question Evaluate the summation for the indicated values of the variable
		$$1(1!)+2(2!)+3(3!)+...+m(m!);m=2$$
		
		\begin{solution}
		$$=1(1!)+2(2!)$$
		$$=1+4$$
		$$=5$$
		\end{solution}
		\setcounter{question}{42} \question Write using summation or product notation
		$$(1-t)\cdot(1-t^2)\cdot(1-t^3)\cdot(1-t^4)$$
		
		\begin{solution}
		$$\prod_{k=1}^{4} (1-t^k)$$
		\end{solution}
		
		
		\setcounter{question}{51} \question Transform by making the change of variable $j=i-1$
		$$\sum^{n-1}_{i=1}\frac{i}{(n-1)^2}$$
		
		\begin{solution}
		$$i=j+1$$
		$$j=i-1$$
		When $i=1$, then $j=1-1=0$ so the lower limit is 0. When $i=n-1$, $j+1=n-1$, $j=n-2$ so the upper limit is $n-2$.\\
		Substituting $i=j+1$ we then have
		$$\sum^{n-1}_{i=1}\frac{i}{(n-i)^2}=\sum^{n-2}_{j=0}\frac{j+1}{(n-j-1)^2}$$
		\end{solution}
		
		\setcounter{question}{54} \question Write as a single summation or product
		$$2\cdot\sum^{n}_{k=1}(3k^2+4)+5\cdot\sum^{n}_{k=1}(2k^2-1)$$
		
		\begin{solution}
		By Theorem 5.1.1 (2)
		$$2\cdot\sum^{n}_{k=1}(3k^2+4)+5\cdot\sum^{n}_{k=1}(2k^2-1)=\sum^{n}_k=(6k^2+8)+\sum^{n}_{k-1}(10k^2-5)$$
				By Theorem 5.1.1 (1)
				$$=\sum^{n}_{k=1}(6k^2+8+10k^2-5)$$
								$$=\sum^{n}_{k=1}16k^2+3$$
		\end{solution}
		
		Compute 59 \& 62. Assume the values of the variables are restricted so the functions are defined.
		\setcounter{question}{58} \question $$\frac{4!}{0!}$$
		
		\begin{solution}
		$0!=1$ by defintion and $4!=4\cdot3\cdot2\cdot1$
		$$=\frac{24}{1}$$
		$$=24$$
		\end{solution}
		
		\setcounter{question}{61} \question $$\frac{n!}{(n-2)!}$$
		
		\begin{solution}
		$$=\frac{n(n-1)(n-2)!}{(n-2)!}$$
		$$=n(n-1)$$
		$$=n^2-n$$
		\end{solution}
		
		\setcounter{question}{73} \question Prove that if $p$ is a prime number and $r$ is an integer with $0<r<p$, then $\binom{p}{r}$ is divisiple by $p$.
		
		\begin{solution}
			\begin{proof}
				Suppose $p$ is a prime number and r is an integer with $0<r<p$\\
				$$\binom{p}{r}=\frac{p!}{r!(p-r)!}$$
				Because $p$ is prime its only factors are $p$ and 1.\\
				$$=p\cdot\frac{(p-1)!}{r!(p-r)!}$$
				Because $r<p$, $(p-r)!$ is contained in $(p-1)!$ so we will get to a point where the fraction becomes $\frac{(p-1)(p-2)(p-r)!}{r!(p-r)!}=\frac{(p-1)(p-2)(...)}{r!}$\\
				From definition of $nCr$ we know that $\binom{p}{r} \in \mathbb{Z}$ and since $p\in\mathbb{Z}$ and since the product of two integers must be an integer then $\frac{(p-1)(p-2)(...)}{r!} \in \mathbb{Z}$\\
				Therefore $\binom{p}{r}=pk$ for some integer $k$ and so by definition $p\mid\binom{p}{r}$
			\end{proof}
		\end{solution}
	\end{questions}
\end{document}