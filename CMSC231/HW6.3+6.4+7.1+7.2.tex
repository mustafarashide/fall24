\documentclass[12pt,letterpaper, onecolumn]{exam}
\usepackage{amsmath,graphicx, mathrsfs, amssymb, amsthm, enumitem, caption, array}

\usepackage[dvipsnames]{xcolor}
\usepackage[lmargin=71pt, tmargin=1.2in]{geometry}
\lhead{Mustafa Rashid\\}
\rhead{Chapters 6.3,6.4,7.1 \& 7,2\\}
\chead{\hline} 
\thispagestyle{empty} 
\newcommand*{\setdef}[1]{\left\{#1 \right\}} 
\newcommand{\doesnotdivide}{\not\hspace{2.5pt}\mid}

\begin{document}
	
	\begingroup  
	\noindent\LARGE Discrete Mathematics\\
	\noindent\LARGE Chapters 6.3,6.4,7.1 \& 7.2 Homework\\
	\noindent\large \today\\
	\noindent\large Mustafa Rashid\par
	\noindent\large Fall 2024\par
	\endgroup
	\rule{\textwidth}{0.4pt}
	\pointsdroppedatright
	\printanswers
	\renewcommand{\solutiontitle}{\noindent\textbf{Ans:}\enspace}  
	
	\centerline{Exercise Set 6.3}\\
	\noindent For 20\& 21 prove the statement that is true and find a counterexample for the statement that is false. Assume all sets are subsets of a universal set U.
\begin{questions}
	\setcounter{question}{19}\question For all sets $A$ and $B, \mathscr{P}(A \cap B)=\mathscr{P}(A)\cap \mathscr{P}(B)$
	\setcounter{question}{20}\question For all sets $A$ and $B, \mathscr{P}(A \times B)=\mathscr{P}(A)\times \mathscr{P}(B)$
	\setcounter{question}{31}\question For all sets $A$ and $B, (A-B)\cup (A\cap B)=A.$ (Construct an algebraic proof and cite a property from Theorem 6.2.2 for every step)
	\setcounter{question}{42}\question $((A\cap (B\cup C))\cap(A-B))\cap (B \cup C^c)$ (Simplify the given expression. Cite a property from
	Theorem 6.2.2 for every step)
\end{questions}
	\centerline{Exercise Set 6.4}

\begin{questions}
		\setcounter{question}{10}\question  Let $S=\{0,1\}$ and define operations + and · on S by the
		following tables:
		\begin{center}
			\includegraphics[width=0.7\linewidth]{6.4q11}
		\end{center}
		\begin{parts}
			\part Show that the elements of $S$ satisfy the following properties:
			\begin{enumerate}[label = (\roman*)]
				\item the commutative law for $+$
				\item the commutative law for $\cdot$
				\item the associative law for $+$
				\item the commutative law for $\cdot$
				\item the distributive law for $+$ over $\cdot$
				\item the distributive law for $\cdot$ over $+$
				\end{enumerate}
			\part Show that 0 is an identity element for $+$ and that 1 is an identity element for $\cdot$.
			\part Define $\overline{0}=1$ and $\overline{1}=0.$ Show that for all $a$ in $S,$ $a+\overline{a}=1$ and $a\cdot\overline{a}=0.$ It follows from parts (a)-(c) that $S$ is a Boolean algebra with operations $+$ and $\cdot$.
		\end{parts}
		\setcounter{question}{11}\question Prove that the associative laws for a Boolean algebra can be omitted from the definition. That is, prove that the associative laws can be derived from the other laws in the definition.
		\setcounter{question}{18}\question   \begin{parts} \part Assuming that the following sentence is a statement, prove that $1+1=3$:
			\begin{center}
				If this sentence is true, then $1+1=3$.
			\end{center}
			\part What can you deduce from part (a) about the status of ``This sentence is true"? Why? (This example is known as \textbf{Löb's paradox.})
		\end{parts}

\end{questions}

	\centerline{Exercise Set 7.1}
\begin{questions}
	\setcounter{question}{13}\question Let $J_5=\{0,1,2,3,4\}$, and define functions $h:J_5\rightarrow J_5$ and $k:J_5 \rightarrow J_5$ as follows: For each $x\in J_5, h(x)=(x+3)^3\mod 5$ and $k(x)=(x^3+4x^2+2x+2)\mod5.$ Is $h=k$? Explain.
	\setcounter{question}{23}\question If $b$ and $y$ are positive real numbers such that $\log_b y=2$, what is $\log_{b^2}(y)$? Why?
	\setcounter{question}{27}\question Student $C$ tries to define a function $h: Q \rightarrow Q$ by the rule 
	$$h\left(\frac{m}{n}\right)=\frac{m^2}{n}, \text{ for all integers $m$ and $n$ with $n\neq0$.}$$
	Student D claims that $h$ is not well defined. Justify student 
	\setcounter{question}{42}\question Given a set $S$ and a subset $A$, the \textbf{characteristic function of A,} denoted $\chi_A,$ is the function defined from $S$ to $\mathbb{Z}$ with the property that for all $u\in S$,
	\begin{equation*}
	\chi_A(u)=
	\begin{cases}
		1 & \text{if } u\in A\\
		0 & \text{if } u\notin A\\
\end{cases}
\end{equation*}
Show that each of the following holds for all subsets $A$ and $B$ of $S$ and all $u\in S.$
\begin{parts}
	\part $\chi_{A\cap B}(u)=\chi_A(u)\cdot\chi_B(u)$
	\part $\chi_{A\cup B}=\chi_A(u)+\chi_B(u)-\chi_A(u)\cdot\chi_B(u)$
\end{parts}
\end{questions}
	\centerline{Exercise Set 7.2}
\begin{questions}
	\setcounter{question}{11}\question \begin{parts}
		\part Define $F:\mathbb{Z}\rightarrow\mathbb{Z}$ by the rule $F(n)=2-3n,$ for all integers $n$.
		\begin{enumerate}[label=(\roman*)]
			\item Is $F$ one-to-one? Prove or give a counterexample.
			\item Is $F$ onto? Prove or give a counterexample.
		\end{enumerate}
		\part Define $G:\mathbb{R}\rightarrow\mathbb{R}$ by the rule $G(x)=2-3x$ for all real numbers $x$. Is $G$ onto? Prove or give a counterexample.
	\end{parts}
	\setcounter{question}{23}\question Define $J:\mathbb{Q}\times\mathbb{Q}\rightarrow\mathbb{R}$ by the rule $J(r,s)=r+\sqrt{2}s$ for all $(r,s)\in\mathbb{Q}\times\mathbb{Q}$.
	\begin{parts}
		\part Is $J$ one-to-one? Prove or give a counter example
		\part Is $J$ onto? Prove or give a counterexample.
	\end{parts}
	\setcounter{question}{30}\question If $f:\mathbb{R}\rightarrow\mathbb{R}$ and $g:\mathbb{R}\rightarrow\mathbb{R}$ are both onto, is $f+g$ also onto? Justify your answer.
		\setcounter{question}{32}\question Let $f:\mathbb{R}\rightarrow\mathbb{R}$ be a function and $c$ a nonzero real number. If $f$ is onto, is $c\cdot f$ also onto? Justify your answer.
\end{questions}
\end{document}