\documentclass[12pt,letterpaper, onecolumn]{exam}
\usepackage{amsmath,graphicx, mathrsfs, amssymb, amsthm, enumitem, caption, array}
\newcommand{\Mod}[1]{\ (\mathrm{mod}\ #1)}
\usepackage[dvipsnames]{xcolor}
\usepackage[lmargin=71pt, tmargin=1.2in]{geometry}
\lhead{Mustafa Rashid\\}
\rhead{Chapters 8.1,8.2,8.3 \& 8.4\\}
\chead{\hline} 
\thispagestyle{empty} 
\newcommand*{\setdef}[1]{\left\{#1 \right\}} 
\newcommand{\doesnotdivide}{\not\hspace{2.5pt}\mid}
\newcommand{\pnt}[1]{{\scriptstyle#1}}

\begin{document}
	
	\begingroup  
	\noindent\LARGE Discrete Mathematics\\
	\noindent\LARGE Chapters 8.1,8.2,8.3 \& 8.4 Homework\\
	\noindent\large \today\\
	\noindent\large Mustafa Rashid\par
	\noindent\large Fall 2024\par
	\endgroup
	\rule{\textwidth}{0.4pt}
	\pointsdroppedatright
	\printanswers
	\renewcommand{\solutiontitle}{\noindent\textbf{Ans:}\enspace}  
	
	\centerline{Exercise Set 8.1}
	\begin{questions}
		\setcounter{question}{19}\question Let $A=\{-1,1,2,4\}$ and $B=\{1,2\}$ and define relations $R$ and $S$ from $A$ to $B$ as follows: For all $(x,y)\in A\times B,$
		\begin{align*}
			x\mathrel{R}y &\Leftrightarrow |x|=|y| \text{ and}\\
			x\mathrel{S}y&\Leftrightarrow x-y \text{ is even}
		\end{align*}
		State explicitly which ordered pairs are in $A\times B, R, S, R\cup S,$ and $R\cap S$. 
		\begin{solution}
			\begin{align*}
				A\times B&=\{(-1,1),(-1,2),(1,1),(1,2),(2,1),(2,2),(4,1),(4,2)\}\\
				R&=\{(-1,1),(1,1),(2,2)\}\\
				S&=\{(-1,1),(1,1),(2,2),(4,2)\}\\
				R\cup S&=\{(-1,1),(1,1),(2,2),(4,2)\}\\
				R\cap S&=\{(-1,1),(1,1),(2,2)\}\\
			\end{align*}
		\end{solution}
	\end{questions}
	\centerline{Exercise Set 8.2}
	\begin{questions}
		\setcounter{question}{13}\question $O$ is the relation defined on $\mathbb{Z}$ as follows: For all $m,n\in\mathbb{Z},$ $m\mathrel{O}n \Leftrightarrow m-n$ is odd. Determine whether this relation is reflexive, symmetric, transitive, or none of these. Justify your answer.
		\begin{solution}\\
			\textbf{O is not reflexive:} Suppose $m$ is a particular but arbitrarily chosen integer. Now $m-m=0$. But this is not odd. Hence $m-m$ is not odd and $O$ is not reflexive.\\
			\textbf{O is symmetric:} Suppose $m$ and $n$ are particular but arbitrarily chosen integers that satisfy the condition $m\mathrel{O}n$. By definition of $O$, since $m\mathrel{O}n$ then $m-n$ is odd. By definition of odd, this means that $m-n=2k+1$ for some integer $k$. Multiplying both sides by $-1$ gives $n-m=2(-k-1)+1$ since $-k-1$ is an integer, this equation shows that $n-m$ is odd. Hence, by defintion of $O$, $n\mathrel{O}m$.\\
			\textbf{O is not transitive:} Suppose $m,n$ and $p$ are particular but arbitrarily chosen integers that satisfy condtion $m\mathrel{O}n$ and $n\mathrel{O}p$. By definition of $O$, since $m\mathrel{O}n$ and $n\mathrel{O}p$, then $m-n$ is odd and $n-p$ is also odd. By definition of odd, this means that $m-n=2k+1$ and $n-p=2q+1$ for some integers $k$ and $q$. Adding the two equations gives $(m-n)+(n-p)=2k+1+2q+1$ and simplifying this gives $m-p=2(k+q+1)$. Since $k+q+1$ this equation shows that $m-p$ is even and not odd. Hence, by defintion of $O$, $m \not \mathrel{O} p$.
		\end{solution}
		\setcounter{question}{35}\question If $R$ is transitive, then $R^{-1}$ is transitive. (Prove or disprove this statement.)
		\begin{solution}
			Suppose $R$ is any relation on  a set $A$ that is transitive. By defintion of transitive, this means that for all $x,y,$ and $z$ in $A$, if $(x,y)\in R$ and $(y,z)\in R$ then $(x,z)\in R$. Suppose that $x\mathrel{R}y$ and  $y\mathrel{R}z$. Because $R$ is transitive, then $x\mathrel{R}z$. The inverse relation by definition would contain the relations $y\mathrel{R^{-1}}x$, $z\mathrel{R^{-1}}y$, and $z\mathrel{R^{-1}}x$. This means that for any $x,y,$ and $z$ in $A$ if $z\mathrel{R^{-1}}y$ and $y\mathrel{R^{-1}}x$ then $z\mathrel{R^{-1}}x$. Hence, $R^{-1}$ is also transitive.
					\end{solution}
	\setcounter{question}{39}\question If $R$ and $S$ are reflexive, is $R\cup S$ reflexive? Why? (Assume $R$ and $S$ are relations on a set $A$. Prove or disprove the statement.)
	\begin{solution}
	Suppose not, suppose that $R$ and $S$ are reflexive and that $R\cup S$ is not reflexive. This means that there is an element $x$ in $R\cup S$ such that $(x,x) \notin R$ or $(x,x)\notin S$. But we know that $R$ and $S$ are reflexive so it is not true that that there is an element $x$ in $R$ such that $(x,x)\notin R$ and it is not true that here is an element $x$ in $S$ such that $(x,x)\notin R$. This means that $R\cup S$ also has to be reflexive because all ordered pairs $(x,x)$ in $R\cup S$ come are in $R$ or $S$ and so they are also reflexive.
	\end{solution}
	\end{questions}
	\pagebreak
	\centerline{Exercise Set 8.3}
		   In 9 and 10 the relation R is an equivalence relation on the set A. Find the distinct equivalence classes of R.\\
	\begin{questions}
		\setcounter{question}{8}\question $X=\{-1,0,1\}$ and $A=\mathscr{P}(X)$. $R$ is defined on $\mathscr{P}(X)$ as follows: For all sets $\pnt{S}$ and $\pnt{T}$ in $\mathscr{P}(X)$,
		\begin{align*}
			\pnt{S}\mathrel{R}\pnt{T}\Leftrightarrow \text{ the sum of elements in $\pnt{S}$ equals the sum of the elements in $\pnt{T}$}
		\end{align*}
		\begin{solution}
			\begin{align*}
			&\{\phi\}\\
			&\{\{-1\},\{-1,0\}\}\\
			&\{\{0\},\{-1,1\},\{-1,0,1\}\}\\
			&\{\{1\},\{0,1\}\}
			\end{align*}
		\end{solution}
		\setcounter{question}{9}\question $A=\{-5,-4,-3,-2,-1,0,1,2,3,4,5\}.$ $R$ is defined on $A$ as follows: For all $m,n\in\mathbb{Z},$
		$$m\mathrel{R}n\Leftrightarrow 3\mid(m^2-n^2).$$
		\begin{solution}
			\begin{align*}
				[0]=[3]=[-3]&=\{-3,0,3\}\\
				[1]=[-1]=[2]=[-2]=[4]=[-4]=[5]=[-5]&=\{-5,-4,-2,-1,1,2,4,5\}
			\end{align*}
		\end{solution}
		\setcounter{question}{38}\question The following argument claims to prove that the requirement that an equivalence relation be reflexive is redundant. In other words, it claims to show that if a relation is symmetric and transitive, then it is reflexive. Find the mistake in the argument.\\
		``\textbf{Proof:} Let $R$ be a relation on a set $A$ and suppose $R$ is symmetric and transitive. For any two elements $x$ and $y$ in $A$, if $x\mathrel{R}y$ then $y\mathrel{R}x$ since $R$ is symmetric. But then it follows by transitivity that $x\mathrel{R}x$. Hence $R$ is reflexive."
		\begin{solution}
			By definiton, a relation $R$ is symmetric if, and only if, for all $x,y\in A$, if $x\mathrel{R}y$ then $y\mathrel{R}x$. However, this is still vacuously true (that is the relation is still symmetric by definition) in the following cases where: $$x\not\mathrel{R}y \lor y\mathrel{R}x$$
			$$x\not\mathrel{R}y \lor y\not\mathrel{R}x$$
			The second case shows that there is an element $x$ in $A$ such that $x\not\mathrel{R}x$ and so $R$ is not necessarily reflexive. 
		\end{solution}
	\end{questions}

	\centerline{Exercise Set 8.4}
	\begin{questions}
		\setcounter{question}{15}\question What is the units digit of $3^{1789}$?
		\begin{solution}
			By looking at $3^0=1, 3^1=3,3^2=9,3^3=27,3^4=81$ we can see that
			\begin{align*}
 				 3^4 &\equiv 1 \Mod{10}
			\end{align*}
			By Theorem 8.4.3 and exponent laws we get
				\begin{align*}
				3^{1788}=(3^4)^{447} &\equiv 1^{447} \Mod{10}\\
				3^{1788}&\equiv1\Mod{10}
			\end{align*}
			We know that $3\equiv1\Mod{10}$ and so by Theorem 8.4.3, $3^{1788}\cdot3\equiv 1\cdot3\Mod{10}$. Therefore, the units digit of $3^{1789}$ is 3.
		\end{solution}
		\setcounter{question}{31}\question Prove that in any commutative ring $R$, there is only one multiplicative identity. In other words, prove that in any commutative ring $R$, if $ec=c$ for all elements $c$ in $R$, then $e=1$
		\begin{solution}
			\begin{proof}
			Suppose $c=1$,and so because $ec=c$ then $e\cdot1=1$ by the identity for multiplication. Because $1$ is the identity for multiplication $e\cdot1=e$ and therfore $e=e\cdot1=1$.
			\end{proof}
		\end{solution}
		\setcounter{question}{32}\question Prove that given any element $a$ in any commutative ring $R$, there is only one additive inverse for $a$. In other words, prove that if $a+b=0$ and $a+c=0$, then $b=c$.
		\begin{solution}
			\begin{proof}
			Because $a+b=0$ and $a+c=0$ we can say that $a+b=a+c$. Subtracting $a$ from both sides gives $b=c$.
			\end{proof}
		\end{solution}
	\end{questions}
	\centerline{Exercise Set 8.5}
	\begin{questions}
			\setcounter{question}{18}\question Use the extended Euclidean algorithm to find the greatest common divisor of 2583 and 349. Express the greatest common divisor as a linear combination of the two numbers.
			\begin{solution}
			\begin{enumerate}
				\item Applying the Euclidean Algorithm
				\begin{enumerate}[label=(\alph*)] 
					\item gcd(2583,349): $2583=348\cdot7+140$
					\item gcd(349,140):$349=140\cdot2+69$
					\item gcd(140,69):$140=69\cdot2+2$
					\item gcd(69,34):$69=34\cdot2+1$
					\item gcd(34,1):$34=1\cdot34+0$
					\item gcd(1,0)$=1$
				\end{enumerate}
				\item Substituting back the results
				\begin{align*}
					1&=69-34\cdot2\tag*{By (d)}\\
					&=69-(140-69\cdot2)34\tag*{By (c)}\\
					&=69\cdoy69-(2583-349\cdot7)\cdot34\tag*{By (a) \& algebra}\\
					&=69(349-140\cdot2)-2583\cdot34+349\cdot7\cdot34\tag*{By (b)}\\
					&=2583\cdot(-172)+349\cdot1273\tag*{By algebra}
				\end{align*}
			\end{enumerate}
			\end{solution}
			\setcounter{question}{40}\question Use the extended Euclidean algorithm to find a postive integer $n$ so that $2\leq n \leq 30$ and $[n]$ is an inverse for $[13]$ in $\mathbb{Z}_{31}$
			\begin{solution}
				$\exists x$ such that $[13]\cdot[x]$ if, and only if, $13x\equiv1\Mod{31}$. So $13x+31(-k)=1$ and applying the euclidean algorithm on 31 and 13 gives:
					\begin{enumerate}[label=(\alph*)] 
					\item gcd(31,13): $31=13\cdot2+5$
					\item gcd(13,5):$13=5\cdot2+3$
					\item gcd(5,2):$5=3\cdot1+2$
					\item gcd(3,2):$3=2\cdot1+1$
					\item gcd(2,1):$2=1\cdot2+0$
					\item gcd(1,0)$=1$
				\end{enumerate}
				Substituting the results back gives:
				\begin{align*}
					1&=3-2\cdot1\\
					&=3-(5-3\cdot1)\cdot1\tag*{By (c)}\\
					&=2(13-5\cdot2)-(31-13\cdot2)\tag{By algebra \& (a)}\\
					&=13\cdot4-31-4(31-13\cdot2)\tag*{By algebra \& (a)}\\
					&=13(12)+31(-5)\\
				\end{align*}
				So $1=13(12)+31(-5)$ or $13\cdot12\equiv1\Mod{31}$ and so $n=12$.
			\end{solution}
	\end{questions}
	
\end{document}