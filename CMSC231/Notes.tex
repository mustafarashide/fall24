\documentclass[12pt,letterpaper, onecolumn]{exam}
\usepackage{amsmath}
\usepackage{amssymb}
\usepackage{amsthm}
\usepackage{graphicx}
\usepackage{caption}
\usepackage{array}
\usepackage[dvipsnames]{xcolor}
\usepackage[lmargin=71pt, tmargin=1.2in]{geometry}
\lhead{Mustafa Rashid\\}
\rhead{Notes\\}
\chead{\hline} 
\thispagestyle{empty} 
\newcommand*{\setdef}[1]{\left\{#1 \right\}} 
\newcommand{\doesnotdivide}{\not\hspace{2.5pt}\mid}

\begin{document}

\begingroup  
\noindent\LARGE Discrete Mathematics\\
\noindent\large \today\\
\noindent\large Mustafa Rashid\par
\noindent\large Fall 2024\par
\endgroup
\rule{\textwidth}{0.4pt}
\pointsdroppedatright
\printanswers
\renewcommand{\solutiontitle}{\noindent\textbf{Ans:}\enspace}  


	\section{General Recursive Definitions and Structural Induction}
	Recursively defined sets 
	\begin{enumerate}
		\item BASE: A statement that certain objects belong to the set.
		\item RECURSION: A collection of rules indicating how to form new set objects from those already known to be in the set.
		\item RESTRICTION: A statement that no objects belong to the set other than those coming from 1 and 2.
	\end{enumerate}
	\textbf{A string over S}: Let S be a finite set with at least one element. A string over S is a finite sequnce of elemnts from S. The elements of S are \textbf{characters} of the string, and the \textbf{length} of a string is the number of characters it contains. The \textbf{null string over S} is defined to be the ``string" with no characters. It is usually denoted $\epsilon$ and is said to have length 0.\\
		\textbf{Structural Induction for Recursively Defined Sets}\\
		Let S be a set that has been defined recursively, and consider a property that objects in S may or may not satisfy. To prove that every object in S satisfies the property:
		\begin{enumerate}
			\item Show that each object in the BASE for S satisfies the property;
			\item Show that for each rule in the RECURSION, if the rule is applied to objects in S that satisfy the property, then the objects defined by the rule also satisfy the property.\\
			NOTE: Because no objects other than those obtained through the BASE and RECURSION conditions are contained in S, it must be the case that every object in S satisfies the property.
			\end{enumerate}
			\textbf{Recursive Function:} A function is said to be defined recursively or to be a recursive function if its rule of definition refers to itself. Because of this self-reference, it is sometimes difficult to tell whether a given recursive function is well defined.
	\section{6.3: Disproofs and Algebraic Proofs}
	Finding counterexamples\\
	Ask:$(A-B)\cup(B-C)=^?A-C$
	\begin{center}
		\includegraphics[width=0.7\linewidth]{findingcounterexamples}
	\end{center}
	There are obvious differences and so the equivalence is not true. For example if $A=\{2,3,5,6\}, B=\{1,2,3,4\}, C=\{3,4,6\}$\\
	What does the venn diagram for four sets look like?
	Is $A\cap C \subseteq B\cup D$? Consider $A=C=\{1\}$ and $B=D=\phi$ so $A\cap C\not\subseteq B\cup D$ and this cannot be the Venn digaram for 4 sets as it doesn't have all the regions we need.
	\begin{center}
		\includegraphics[width=0.7\linewidth]{wrong4setvenndiagram}
	\end{center}
	The Venn diagram for four sets with all regions present
	\begin{center}
		\includegraphics[width=0.7\linewidth]{correct4setdiagram}
	\end{center}
	$A \cap C =\{8,\textbf{10},12,13\}$\\
	$B \cup D =\{1,4,5,6,7,8,9,11,12,13,14,15\}$\\
	$A \cap C \not\subseteq B \cup D$\\
	$2^n$ regions $(2^{n-1})$ for $n$ sets\\
	\textbf{Algebraic Proofs}\\
	Is $(A\cup B)-C=(A-C)\cup (B-C)$?
	\begin{align*}
		(A\cup B)&=(A\cup B)\cap C^c\tag*{By definition of minus}\\
		&= C^c \cap (A\cup B) \tag*{By commutativity}\\
		&=(A\cap C^c)\cup(C^c\cap B)\tag*{By distribution}\\
		&=(A\cap C^c)\cup(B\cap C^c)\tag*{By commutativity}\\
		&=(A-C)\cup(B-C)\tag*{By definition of minus}
	\end{align*}
	\section{6.4: Boolean \& Russel's Paradox}
	\textbf{Boolean Algebra:}\\
	Is a set $B$ with operations '+' and '$\cdot$' with the following properties:\\
	0. Closure: $\forall x,y \in B, x+y\in B \land x\cdot y \in B$\\
	1. Commutavity: $\forall x,y \in \B, x+y=y+x \land x\cdot y=y\cdot x$\\
	2. Associativity: $\forall x,y,z \in B, (x+y)+z=x+(y+z) \land (x\cdot y)\cdot z=x\cdot(y\cdot z)$\\
	3. Distributivity: $\forall x,y,z \in B, x+(y\cdot z)=(x+y)\cdot(x+z) \land x(y+z)=(x\cdot y)+(x\cdot z)$\\
	4. Identity: $\exists 0,1 \in B$ such that $\forall x \in B, x+0=x \land x\cdot1=x$ (0 is false like thing, 1 is true, + corresponds to or, $\cdot$ corresponds to and)\\
	5. Complement: $\forall x \in B, \exists \overline{x}$ such that $x+\overline{x}=1$ and $x\cdot\overline{x}=0$\\
	\textbf{Canonical examples of boolean algebra}
	\begin{itemize}
		\item $S=\{\text{propositional formulae}\}$, 0=false, 1=true, + is $\lor$, $\cdot$ is $\land$, $\overline{x}=\neg x$
		\item $S=\{\text{sets}\}, 0=\phi, 1=S, +$ is $\cup$, $\cdot$ is $\cap$, $\overline{x}=x^c$.
	\end{itemize}
		\textbf{Russel's Paradox:}\\
		$S=\{A:A \text{ is a set and $A\notin A$}\}$\\
		$S\in S:$ suppose $S\in S$. Then $S$ is a set $S\notin S$ which is a contradiction. Alternatively, Suppose $S\notin s$. Then either $S$ is not a set, or $S\in S$. If $S$ is a set:$\Rightarrow\!\Leftarrow$
	\section{7.1-7.2: Functions}
	A function $f$ from domain $X$ to codomain $Y (f:x \rightarrow y)$ is a total, single-valued relation $(x,y)\in f, f(x)=y$ the latter can be read ($f$ of $x$ is $y$, the value of $f$ at $x$ is y, the output of $f$ for input $x$ is $y$, the image of $x$ under $f$ is $y$)\\
	Range: $f(X)=\{f(x):x\in X\}$ - the image of the domain 
	Image of set under $f$\\
	$$f(A)=\{f(x):x\in A\}$$
	Pre-image (also known as Inverse Image)
	$$f^{-1}(C)=\{x\in X:f(x)\in C\}$$
	$$f^{-1}(y)=f^{-1}(\{y\})$$
	Remember: A function is not well defined if it is not total or signle-valued. Example $f(x)=x^2$ then $f^-1$ is not a well-defined function.\\
	New terminology:
	\begin{enumerate}
		\item Injective (one-to-one:inverse is single valued): A function is injective if $f(x)=f(y)\Leftrightarrow x=y$
		\item Surjective (onto: inverse is total): $\forall y\in Y, \exists x \in X$ such that $f(x)=y$.
		\item Bijective (one-to one correspondence): Injective and surjective meaning that the inverse is a well-defined function
		\item Identity function: $$I_x:X\rightarrow X$$ $$I_x(a)=a\hspace*{0.25cm},\forall a\in X$$
	\end{enumerate}
	Logarithm:\\
	$$\log_b:\mathbb{R^+}\rightarrow \mathbb{R}$$
	$$\log_b x=y \Leftrightarrow b^y=x$$
	$$\log_b (xy)=log_b x+log_b y$$
	$$\log_b x^y=y\cdot log_b x$$
	$$\log_b \left(\frac{x}{y}\right)=\log_b(xy^-1)=\log_b x+\log_b y^-1=\log_b x-\log_b y$$
	$$$$
\end{document}