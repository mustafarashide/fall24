\documentclass[12pt,letterpaper, onecolumn]{exam}
\usepackage{amsmath}
\usepackage{amssymb}
\usepackage{amsthm}
\usepackage{graphicx}
\usepackage{caption}
\usepackage{array}
\usepackage[dvipsnames]{xcolor}
\usepackage[lmargin=71pt, tmargin=1.2in]{geometry}
\lhead{Mustafa Rashid\\}
\rhead{Chapters 5.4 \& Supplement A\\}
\chead{\hline} 
\thispagestyle{empty} 
\newcommand*{\setdef}[1]{\left\{#1 \right\}} 
\newcommand{\doesnotdivide}{\not\hspace{2.5pt}\mid}

\begin{document}
	
	\begingroup  
	\noindent\LARGE Discrete Mathematics\\
	\noindent\LARGE Chapters 5.4 \& Supplement A\\
	\noindent\large \today\\
	\noindent\large Mustafa Rashid\par
	\noindent\large Fall 2024\par
	\endgroup
	\rule{\textwidth}{0.4pt}
	\pointsdroppedatright
	\printanswers
	\renewcommand{\solutiontitle}{\noindent\textbf{Ans:}\enspace}  
	
	\centering\large Exercise Set 5.4\\
	\begin{questions}
			\setcounter{question}{1} \question  Suppose $b_1,b_2,b_3,...$ is a sequence defined as follows:
			\begin{align*} 
				b_1=&4, b_2=12\\
				b_k=&b_{k-2}+b_{k-1}\textrm{  for all integers } k\geq3
			\end{align*}
			Prove that $b_n$ is divisible by 4 for all integers $n\geq1.$
			\begin{solution}
				\begin{proof}
				Let the property $P(n)$ be the sentence
				\begin{quote}
						$b_n$ is divisible by 4
				\end{quote}
				\textbf{Show that $P(1)$ and $P(2)$ are true:}\\
				We know that $b_1=4$ and $b_2=12$. Because 4 and 12 are divisible by 4 we know that $P(1)$ and $P(2)$ are true.\\
				\textbf{Show that for all integers $k\geq2$, if $P(i)$ is true for all integers from 1 to through $k$, then $P(k+1)$ is also true.}\\
				Let $k$ be any integer with $k\geq2$, and suppose that $b_i$ is divisible for all integers $i$ with $1\leq i\leq k$. We must show that $k+1$ is divisible by 4. We know that $b_{k+1}=b_{k-1}+b_k$ by definition of  $b_1,b_2,b_3,...$\\
				Since $k-1\leq k$ and $k\leq k$, by our inductive hypothesis $b_{k-1}$ and $b_k$ are both divisible by 4. By definition  $b_{k-1}=4q$ for some integer $q$ and $b_k=4s$ for some integer $s$ so $b_{k+1}=4q+4s=4(q+s)$ and so $b_{k+1}$ is divisible by 4.
				\end{proof}
			\end{solution}
		\setcounter{question}{6} \question  Suppose $g_1,g_2,g_3,...$ is a sequence defined as follows:
		\begin{align*} 
			g_1=&3, g_2=5\\
			g_k=&3g_{k-1}-2g_{k-2}\textrm{  for all integers } k\geq3
		\end{align*}
					Prove that $g_n=2^n+1$ all integers $n\geq1.$
		\begin{solution}
			\begin{proof}
			Let  $g_1,g_2,g_3,...$ be the sequence defined by specifiying that $g_1=3$, $g_2=5$, and $g_k=3g_{k-1}-2g_{k-2}$ for all integers $k\geq3$, and let the property $P(n)$ be the formula
			$$g_{n}=2^n+1$$
			We will use strong mathematical induction to prove that for all integers $n\geq1$, $P(n)$ is true.\\
			\textbf{Show that $P(1)$ and $P(2)$ are true:}\\
			To establish $P(1)$ and $P(2)$, we must show that\\
			$$g_1=2^1+1 \textrm{ and } g_2=2^2+1$$
			But, by definition of $g_1,g_2,g_3,....$, we must have that $g_1=3$ and $g_2=5$. Since $2^1+1=2+1=3$ and $2^2+1=4+1=5$, the values of $g_1$ and $g_2$ agree with the values given by the formula.
			\texbf{Show that for all integers $k\geq2$, if $P(i)$ is true for all integers i from 1 through k, then $P(k+1)$ is also true: }\\
			Let $k$ be any integer with $k\geq2$ and suppose that
			$$g_i=2^i+1 \textrm{ for all integers $i$ with } 1\leq i \leq k$$
			We must show that 
			$$g_{k+1}=2^{k+1}+1$$
			But since $k\geq2$, we have that $k+1 \geq3$, and so 
			\begin{align*}
				g_{k+1}&=3g_k-2g_{k-1}\\
				&=3(2^k+1)-2(2^{k-1}+1)\\
				&=3\cdot2^k+3-2^k-2\\
				&=3\cdot2^k-2^k+1\\
				&=(1+2)2^k-2^k+1\\
				&=2^k+2^{k+1}-2^k+1\\
				&=2^{k+1}+1
			\end{align*}
			\end{proof}
	\end{solution}
		\setcounter{question}{8} \question Define a sequence $a_1,a_2,a_3,...$ as follows: $a_1=1,a_2=3,$ and $a_k=a_{k-1}+a_{k-2}$ for all integers $k\geq3$. Use strong mathematical induction to prove that $a_n \leq \left(\frac{7}{4}\right)^n$ for all integers $n\geq1.$
		\begin{solution}
			\begin{proof}
			Let $a_1,a_2,a_3,...$  be the sequence defined by specifying that $a_1=1, a_2=3$ and $a_k=a_{k-1}+a_{k+2}$ for all integers $k\geq3$, and let the property $P(n)$ be the inequality
			$$a_n\leq \left(\frac{7}{4}\right)^n$$
			We will use strong mathematical induction to prove that for all integers $n\geq1$, $P(n)$ is true.\\
			\textbf{Show that $P(1)$ and $P(2)$ are true:}\\
			To establish $P(1)$ and $P(2)$ we must show that\\
			$$a_1\leq \left(\frac{7}{4}\right)^1 \textrm{ and }a_2\leq \left(\frac{7}{4}\right)^2$$
		But by definition of $a_1,a_2,a_3,...$, we have that $a_1=1$ and $a_2=3$. Since $1<\frac{7}{4}$ and $3<(\frac{7}{4})^2$, the inequality holds for $a_1$ and $a_2$.\\
		\textbf{Show that for all integers $k\geq2$, if $P(i)$ is true for all integers from 1 through $k$, then $P(k+1)$ is also true:}\\
		Let $k$ be any integer with $k\geq1$ and suppose that 
		$$a_i \leq \left(\frac{7}{4}\right)^i \textrm{ for all integers } i \textrm{ with } 1\leq i \leq k$$
		We must show that
		$$a_{k+1}\leq \left(\frac{7}{4}\right)^{k+1}$$
		But since $k\geq2$, we have $k+1 \geq 3$, and so
		$$a_{k+1}=a_k+a_{k-1}\leq \left(\frac{7}{4}\right)^k + \left(\frac{7}{4}\right)^{k-1} $$
		$$\left(\frac{7}{4}\right)^k + \left(\frac{7}{4}\right)^{k-1} \leq \left(\frac{7}{4}\right)^{k+1}$$
		$$a_{k+1}\leq \left(\frac{7}{4}\right)^{k+1}$$
			\end{proof}
		\end{solution}
		\pagebreak
		\setcounter{question}{17} \question Compute $9^0,9^1,9^2,9^3,9^4,$ and $9^5$. Make a conjecture about the units digit of $9^n$ where $n$ is a positive integer. Use strong mathematical induction to prove your conjecture.
		\begin{solution}
		\begin{center}
	\begin{tabular}{c | c}
	$9^0$&1\\
	$9^1$&9\\
	$9^2$&81\\
	$9^3$&729\\
	$9^4$&6561\\
	$9^5$&59049\\
\end{tabular}\\
		\end{center}
	
		\textbf{Conjecture: }The units digit of $9^n$ equals 1 if $n$ is even and equals 9 if $n$ is odd.
		\begin{proof}[Proof by strong mathematical induction:]
		 Let the property $P(n)$ be the sentence
		\begin{quote}
			The units digit of $9^n$ equals 1 if $n$ is even and equals 9 if $n$ is odd.
		\end{quote}
		\textbf{Show that $P(1)$ and $P(2)$ are true:}\\
		When $n=1, 9^n=9^1=9$, and the units digit is 9. When $n=2$, then $9^n=9^2=81$, and the units digits is 1. Thus $P(1)$ and $P(2)$ are true.\\
		\textbf{Show that for any integer $k\geq2$, if the property is true for all integers $i$ with $0\leq i \leq k$ then it is true for $k+1$:}\\
		Let $k$ be any integer with $k\geq2$, and suppose that for all integers $i$ with $0\leq i \leq k$, the units digit of $9^i$ equals 1 if $i$ is even and equals 9 if $i$ is odd. We must show that the units digit of $9^{k+1}$ equals 1 if $k+1$ is even and equals 9 if $k+1$ is odd.\\
		\textbf{Case 1 ($k+1$ is even):}\\
		In this case, $k$ is odd, and so, by inductive hypothesis, the units digit of $9^k$ is 9. Thus $9^k=10q+9$ for some nonnegative integer $q$. It follows that $9^{k+1}=9^k\cdot 9=(10q+9)\cdot 9=90q+81=10(9q+8)+1$. Thus the units digit of $9^{k+1}$ is 1.\\
		\textbf{Case 2 ($k+1$ is odd):}\\
		In this case, $k$ is even, and so, by inductive hypothesis, the units digit of $9^k$ is 1. Thus $9^k=10q+1$ for some nonnegative integer $q$. It follows that $9^{k+1}=9^k\cdot 9=(10q+1)\cdot 9=90q+9=10(9q)+9$. Thus the units digit of $9^{k+1}$ is 9.\\
		\end{proof}
		\end{solution}
		\question Find the mistake in the following ``proof" that purports to show that every nonnegative integer power of every nonzero real number is 1.\\
		\textbf{``Proof:} Let $r$ be any nonzero real number and let the property $P(n)$ be the equation $r^n=1.$\\
		\textbf{Show that $P(0)$ is true:} P(0) is true because $r^0=1$ by defintion of zeroth power.\\
		\textbf{Show that for all integers $k\geq0$, if $P(i)$ is true for all integers $i$ from 0 through $k$, then $P(k+1)$ is also true:} Let k be any integer with $k\geq0$ and suppose that $r^i=1$ for all integers $i$ from 0 through $k$. This is the inductive hypothesis. We must show that $r^{k+1}=1.$ Now\\
		\begin{align*}
			r^{k+1}&=r^{k+k-(k-1)}\tag{$k+k-(k-1) = k+k-k+1 = k+1$}\\
			&=\frac{r^k\cdot r^k}{r^{k-1}}\tag{by the laws of exponents}\\
			&=\frac{1\cdot1}{1}\tag{by inductive hypothesis}\\
			&=1.
		\end{align*}
	Thus $r^{k+1}=1$ \textit{[as was to be shown].}\\
	\textit{[Since we have proved the basis step and the inductive step of the strong mathematical induction, we conclude that the given statement is true.]"}
	\setcounter{question}{20} \question Use the well-ordering principle for the integers to prove the existence part of the unique factorization of integers theorem: Every integer greater than 1 is either prime or a product of prime numbers.
	\begin{solution}
		It is possible to think of examples where the basis step is not true. For instance, $P(1)$ is false because $r^1=r$. This is only true when $r=1$ and it is false for every other real number.
	\end{solution}
	\setcounter{question}{25} \question Suppose $P(n)$ is a property such that
	\begin{enumerate}
	\item $P(0), P(1), P(2)$ are all true,
	\item for all integers $k\geq0$, if $P(k)$ is true, then $P(3k)$ is true. Must it follow that $P(n)$ is true for all integers $n\geq0$? If yes, explain why; if no, give a counterexample.
	\end{enumerate}
	\begin{solution}
		No, $P(n)$ is not necessarily true for all integers. Consider the property $P(n)$ which represents the statement ``$n$ is less than 3". $P(3n)$ is not true for $n=1$ or $n=2$.
	\end{solution}
	\setcounter{question}{31}\question Prove that if a statement can be proved by ordinary mathematical induction, then it can be proved by the well-ordering principle.
	\begin{solution}
		\begin{proof}
		Suppose not. That is suppose there is a statement the form ``For all integers $n\geq a$, a property $P(n)$ is true.” that can be proved by ordinary mathematical induction and that can not be proved by the well-ordering principle. Because the statement can be proved by ordinary mathematical indudction we know the following:
		\begin{enumerate}
			\item $P(a)$ is true.
			\item For all integers $k\geq a$, if $P(k)$ is true then $P(k+1)$ is true.
		\end{enumerate}
		Then, suppose that the set $S$ is the set of all integers greater than or equal to a for which $P(n)$ is false. Because of our supposition the set $S$ does not have a least element but because our statement is proved by ordinary mathematical induction then the set $S$ must be empty, and so the well-ordering principle is not violated and this is a contradiction of our supposition.
		\end{proof}
	\end{solution}
	\end{questions}	
	
	\centering\large Supplement A\\
\begin{questions}
	\setcounter{question}{1} \question Show that if the predicate is true before entry to the loop, then it is also true after exit from the loop
		\begin{align*}
		&\textrm{loop: }\textbf{while}(m\geq0)\textrm{ and }m\leq100)\\
		&\hspace*{1.5cm}m:=m+4\\
		&\hspace*{1.5cm}n:=n-2\\
		&\hspace*{1.25cm}\textbf{end while}\\
		&\textrm{predicate: }m+n \textrm{ is odd}\\
	\end{align*}
	\begin{solution}
		\begin{proof}
		Suppose the predicate $m+n$ is odd is true before entry to the loop. Then
		$$m_{old}+n_{old}=2q+1 \textrm{ for some integer $q$}$$
		After execution of the loop,
		$$m_{new}=m_{old}+4 \textrm{ and } n_{new}=n_{old}-2$$
		so
		\begin{align*}
		m_{new}+n_{new}&=m_{old}+4+n_{old}-2\\
		&=m_{old}+n_{old}+2=2q+1+2\\
		&=2(q+1)+1
		\end{align*}
		Therefore $m_{new}+n_{new}=2(q+1)+1$ and is thus odd by definition.
		\end{proof}
	\end{solution}
	Exercises 7 contains a while loop annotated with a pre- and a post-condition and also a loop invariant. Use the loop invariant theorem to prove the correctness of the loop with respect to the pre- and post-conditions.
	\setcounter{question}{6} \question \textit{[Pre-condition: largest = A[1] and i=1]}\\
	\hspace*{0.25cm} \textbf{while} $(i\neq m)$\\
	\hspace*{0.75cm} 1. $i:=i+1$\\
	\hspace*{0.75cm} 2. \textbf{if} $A[i]>$ largest \textbf{then} largest: $=A[i]$\\
	\hspace*{0.25cm} \textbf{end while}\\
	\textit{[Post-condition: largest = maximum value of A[1], A[2],...,A[m]]}\\
	loop invariant: $I(n)$ is ``largest = maximum value of $A[1], A[2],A[n+1]$ and $i=n+1$”
	\begin{solution}\\
		\textbf{I. Basis Property}\\
		$I(0)$ is largest = maximum value of  $A[1]$ and $i=1$. According to the pre-condition the largest = A[1] and $i=1$. Therefore $I(0)$ is true.\\
		\textbf{II. Inductive Property}\\
		Suppose $k$ is any nonnegative integer such that $G\land I(k)$ is true before iteration of the loop. Then as execution reaches the top of the loop, $i\neq m$, largest =$A[k]$, and $i=k+1$. Since $i\neq m$, the guard is passed and statement 1 is executed. Now before execution of statement 1,
		$$i_{old}=k+1$$
		so execution of statement 1 has the following effect:
		$$i_{new}=i_{old}+1$$
		Similarly, before execution of statement 2
		$$largest_{old}=\textrm{ maximum value of } A[1], A[2],...,A[k+1]$$
		so after exeuction of statement 2,
		$$largest_{new}=\textrm{ maximum value of } A[1], A[2],...,A[k+2]$$
		Hence after loop iteration, the two statements $i=k+2$ and largest = maximum value of  $A[1], A[2],...,A[k+2]$ are true, and so $I(k+1)$ is true.\\
		\textbf{III. Eventual Falsity of Guard}\\
		The guard G is the condition $i\neq m$, and $m$ is a nonnegative integer. By I and II, it is known that
		\begin{quote}
			for every integer $n\geq0$, if the loop is iterated $n$ times then largest = maximum value of $A[1], A[2],...,A[n+1]$ and $i=n+1$ 
		\end{quote}
		So after $m$ iterations of the loop, $i=m$. Thus $G$ becomes false after $m$ iterations of the loop\\
		\textbf{IV. Correctness of the Post-Condition}\\
		According to the post-condition, the value of largest after execution of the loop should be the maximum of $A[1], A[2],...,A[m]$. But when $G$ is false $i=m$. And when $I(N)$ is true, $i=N+1$ and largest = maximum value of $A[1], A[2],...,A[N+1]$. Since both conditions are satisfied, $m=i=N+1$ and largest = maximum value of $A[1], A[2],...,A[m]$ as required.
	\end{solution}
\end{questions}
\end{document}