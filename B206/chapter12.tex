\documentclass[12pt,letterpaper, onecolumn]{exam}
\usepackage{amsmath,graphicx}
\usepackage{amssymb}
\usepackage{amsthm}
\usepackage{enumitem}
\usepackage{caption}
\usepackage{array}
\usepackage{mathrsfs}
\usepackage[dvipsnames]{xcolor}
\usepackage[lmargin=71pt, tmargin=1.2in]{geometry}
\lhead{Mustafa Rashid\\}
\rhead{Chapter 11\\}
\chead{\hline} 
\thispagestyle{empty} 
\newcommand*{\setdef}[1]{\left\{#1 \right\}} 
\newcommand{\doesnotdivide}{\not\hspace{2.5pt}\mid}

\newtheorem*{prop}{Proposition}

\begin{document}
	\begingroup  
	\noindent\LARGE B206 | Transitions to Higher Maths\\
	\noindent\LARGE Chapter 12\\
	\noindent\large \today\\
	\noindent\large Mustafa Rashid\par
	\noindent\large Fall 2024\par
	\endgroup
	\rule{\textwidth}{0.4pt}
	\pointsdroppedatright
	\printanswers
	\renewcommand{\solutiontitle}{\noindent\textbf{Ans:}\enspace}  
	\centerline{Section 12.2}
	\begin{questions}
		\setcounter{question}{9}\question Prove the function $f:\mathbb{R}-\{1\}\rightarrow \mathbb{R}-\{1\}$ defined by $f(x)=\left(\frac{x+1}{x-1}\right)^3$ is bijective.
	\end{questions}	
	\centerline{Section 12.6}
	\begin{questions}
		\setcounter{question}{4}\question Consider the function $f:A\rightarrow B$ and a subset $X\subseteq A.$ We observed in Example 12.14 that $f^{-1}(f(X))\neq X$ in general. However, $X\subseteq f^{-1}(f(X))$ is always true. Prove this.
	\end{questions}
\end{document}	