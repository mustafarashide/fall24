\documentclass[12pt,fleqn]{article}
\usepackage{graphicx} % Required for inserting images
\usepackage{amsthm}
\usepackage{amsmath}
\usepackage{amssymb}
\usepackage{mathrsfs}
\usepackage{dsfont}

\newtheorem*{prop}{Proposition}
\newcommand{\doesnotdivide}{\not\hspace{2.5pt}\mid}

\textheight 9in
\textwidth 6.5in
\voffset -0.85in
\hoffset -0.85in
\begin{document}
	
	
	\noindent Chapter 5 (Sep. 9)\\
	Indirect proof of conditional statements
	$$p \longrightarrow q$$
	1. Proof by contradiction\\
	2. Contra-positive Proof\\
	\begin{prop}
		If $P$, then $Q$
	\end{prop}
	\begin{proof}
		Suppose $\neg Q$\\
		(details)\\
		Therefore $\neg P$
	\end{proof}
	
	\begin{tabular}{c c c c c c}
		P&Q&$P\Longrightarrow Q$&$\neg Q$&$\neg P$&$\neg Q \Longrightarrow \neg P$  \\
		T&T&T&F&F&T\\
		T&F&F&T&F&F\\
		F&T&T&F&T&T\\
		F&F&T&T&T&T\\
	\end{tabular}\\
	\\
	Chapter 5, Exercise 2\\
	$$\textrm{Suppose } x\in\mathbb{Z}, \textrm{If }x^2 \textrm{ is odd, then }x \textrm{ is odd} $$
	Two options:\\
	1. Direct proof\\
	$$x^2 \textrm{ is odd } (P) \Longrightarrow$$ 
	$$x^2=2a+1, a \in\mathbb{Z} \Longrightarrow$$ 
	$$?$$
	$$x=2b+1, b\in \mathbb{Z} \Longrightarrow$$
	$$x \textrm{ is odd } (Q)$$\\
	
	2. Indirect proof\\
	$$x \textrm{ is even } (\neg Q) \Longrightarrow$$ 
	$$x=2a, a\in \mathbb{Z} \Longrightarrow$$ 
	$$x^2=(2a)^2=4a^2=2(2a^2)$$ 
	$$b=2a^2, b\in \mathbb{Z} \Longrightarrow$$
	$$x^2 \textrm{ is even }(\neg P)$$
	
	\begin{proof}
		We suppose the contra-positive of the given statement. This means we suppose $x$ is not odd, and we argue that $x^2$ is not odd. So suppose $x$ is even. Then $x=2a$ for some $a\in \mathbb{Z}$ by definition of even. This means 
		$$x^2=(2a)^2=4a^2=2(2a^2).$$
		Let $b=2a$. Then $x^2=2b$, where $b\in\mathbb{Z},$ so we can conclude that $x^2$ is even by definition. 
	\end{proof}
	
	Chapter 5, Exercise 4\\
	$$\textrm{Suppose }a,b,c \in \mathbb{Z}. \textrm{ If $a$ does not divide $bc$, then $a$ does not divide b.}$$
	Recall
	$$a\mid b \Leftrightarrow b=a\cdot c,  a\in \mathbb{Z}$$
	Two options:\\
	1. Direct proof\\
	$$a \doesnotdivide c\textrm{ }(P) \Rightarrow$$
	$$?$$
	$$a \doesnotdivide b\textrm{ }(Q)$$\\
	2. Indirect proof:\\
	$$a\mid b\textrm{ }(\neg Q) \Rightarrow$$
	$$b=ad, d \in \mathbb{Z} \Rightarrow$$
	$$bc=(a\cdot d)\cdot c =a\cdot(d\cdot c)$$
	$$e=d\cdot c, e\in \mathbb{Z} \Rightarrow$$
	$$a\mid bc\textrm{ }(\neg P)$$
	Definition of congruent modulo n
	$$a\equiv b \Leftrightarrow n\mid a-b$$
	$$\Leftrightarrow a-b=n\cdot c, c\in \mathbb{Z}$$
	$$\Leftrightarrow \textrm{$a$ and $b$ have the same remainder when divided by $n$}$$
	\noindent Chapter 8 (Oct. 9) - Proofs involving sets\\
	\textbf{Example 8.8:} Prove that if A and B are sets then $\mathscr{P}(A) \cup \mathscr{P}(B)\subseteq \mathscr{P}(A\cup B)$
	\begin{proof}
		$$X \in \mathscr{P}(A) \cup \mathscr{P}(B) \Longrightarrow$$
		$$X\in \mathscr{P}(A)  \textrm{ or } \mathscr{P}(B) \Longrightarrow$$
		$$X\subseteq A \textrm{ or } X\subseteq B \Longrightarrow$$
		$$X \subseteq A \cup B \Longrightarrow$$
		$$X \in \mathscr{P}(A \cup B)$$
		$$\mathscr{P}(A) \cup \mathscr{P}(B) \subseteq \mathscr{P}(A \cup B)$$
	\end{proof}
	\textbf{Example 8.9:} Suppose A and B are sets. If $\mathscr{P}(A) \subseteq \mathscr{P}(B),$ then $A\subseteq B.$
	(Page 162)\\
	\bigskip
	How do we get proofs with aligned equations on \LaTeX (like Example 8.13)?
	\begin{proof}
		\setlength{\mathindent}{5mm}
	\begin{align*}
	2x-5y&=y\\
	3x+9y&=3\\
\end{align*}
	\end{proof}
	

	
	
\end{document}
