\documentclass[12pt,letterpaper, onecolumn]{exam}
\usepackage{amsmath,graphicx}
\usepackage{amssymb}
\usepackage{amsthm}
\usepackage{enumitem}
\usepackage{caption}
\usepackage{array}
\usepackage[dvipsnames]{xcolor}
\usepackage[lmargin=71pt, tmargin=1.2in]{geometry}
\lhead{Mustafa Rashid\\}
\rhead{Chapter 10\\}
\chead{\hline} 
\thispagestyle{empty} 
\newcommand*{\setdef}[1]{\left\{#1 \right\}} 
\newcommand{\doesnotdivide}{\not\hspace{2.5pt}\mid}

\newtheorem*{prop}{Proposition}

\begin{document}
	\begingroup  
	\noindent\LARGE B206 | Transitions to Higher Maths\\
	\noindent\LARGE Chapter 10\\
	\noindent\large \today\\
	\noindent\large Mustafa Rashid\par
	\noindent\large Fall 2024\par
	\endgroup
	\rule{\textwidth}{0.4pt}
	\pointsdroppedatright
	\printanswers
	\renewcommand{\solutiontitle}{\noindent\textbf{Ans:}\enspace}  
	\begin{questions}
			\setcounter{question}{1}\question Prove that $1^2+2^2+3^2+4^2+...+n^2=\frac{n(n+1)(2n+1)}{6}$ for every positive integer $n$.
			\begin{solution}
				\begin{prop}
					If $n\in\mathbb{N}$ then  $1^2+2^2+3^2+4^2+...+n^2=\frac{n(n+1)(2n+1)}{6}$.
					\end{prop}
					\begin{proof}
					(1) - If $n=1$, this statement is $1^2=\frac{1(2)(3)}{6}$ or $1=1$ which is true.\\
					(2) - We must prove $S_k\Rightarrow S_{k+1}$ for any $k\geq1$. That is we must show that if $1^2+2^2+3^2+4^2+...+k^2=\frac{k(k+1)(2k+1)}{6}$ then $1^2+2^2+3^2+4^2+...+(k+1)^2=\frac{(k+1)(k+1+1)(2(k+1)+1)}{6}$. We use direct proof. Suppose $1^2+2^2+3^2+4^2+...+k^2=\frac{k(k+1)(2k+1)}{6}$. Then
					\begin{align*}
						1^2+2^2+3^2+4^2+...+(k+1)^2&=\\
						1^2+2^2+3^2+4^2+...+k^2+(k+1)^2&=\\
						\frac{k(k+1)(2k+1)}{6}+(k+1)^2&=\\
						\frac{k(k+1)(2k+1)+6(k+1)(k+1)}{6}&=\\
						&=\frac{(k+1)\left[k(2k+1)+6(k+1)\right]}{6}\\
						&=\frac{(k+1)(2k+3)(k+2)}{6}\\
						&=\frac{(k+1)(k+1+1)(2(k+1)+1)}{6}
					\end{align*}
					Thus $1^2+2^2+3^2+4^2+...+(k+1)^2=\frac{(k+1)(k+1+1)(2(k+1)+1)}{6}$. This proves that $S_k\Rightarrow S_{k+1}$. It follows by induction that $1^2+2^2+3^2+4^2+...+n^2=\frac{n(n+1)(2n+1)}{6}$ for every positive integer $n$.
					\end{proof}
			\end{solution}
			\pagebreak
			\setcounter{question}{7}\question If $n\in\mathbb{N},$ then $\frac{1}{2!}+\frac{2}{3!}+\frac{3}{4!}+...+\frac{n}{(n+1)!}=1-\frac{1}{(n+1)!}$.
			\begin{solution}
				\begin{prop}
					If $n\in\mathbb{N}$, then $\frac{1}{2!}+\frac{2}{3!}+\frac{3}{4!}+...+\frac{n}{(n+1)!}=1-\frac{1}{(n+1)!}$.
				\end{prop}
				\begin{proof}
					(1) - If $n=1$, this statement is $\frac{1}{2!}=1-\frac{1}{2!}$ or $\frac{1}{2}=\frac{1}{2}$ which is true.\\
					(2) - We must prove $S_k\Rightarrow S_{k+1}$ for any $k\geq1$. That is we must show that if $\frac{1}{2!}+\frac{2}{3!}+\frac{3}{4!}+...+\frac{k}{(k+1)!}=1-\frac{1}{(k+1)!}$ then $\frac{1}{2!}+\frac{2}{3!}+\frac{3}{4!}+...+\frac{k+1}{(k+1+1)!}=1-\frac{1}{(k+1+1)!}$. We use direct proof. Suppose $\frac{1}{2!}+\frac{2}{3!}+\frac{3}{4!}+...+\frac{k}{(k+1)!}=1-\frac{1}{(k+1)!}$. Then
					\begin{align*}
					\frac{1}{2!}+\frac{2}{3!}+\frac{3}{4!}+...+\frac{k+1}{(k+1+1)!}&=\\
					\frac{1}{2!}+\frac{2}{3!}+\frac{3}{4!}+...+\frac{k}{(k+1)!}+\frac{k+1}{(k+1+1)!}&=\\
					1-\frac{1}{(k+1)!}+\frac{k+1}{(k+1+1)!}&=1-\frac{(k+2)+k+1}{(k+1+1)!}\\
					&=1-\frac{1}{(k+1+1)!}
					\end{align*}
					Thus $\frac{1}{2!}+\frac{2}{3!}+\frac{3}{4!}+...+\frac{k+1}{(k+1+1)!}=1-\frac{1}{(k+1+1)!}$. This proves that $S_k\Rightarrow S_{k+1}$. It follows by induction that $\frac{1}{2!}+\frac{2}{3!}+\frac{3}{4!}+...+\frac{n}{(n+1)!}=1-\frac{1}{(n+1)!}$ for every positive integer n.
				\end{proof}
			\end{solution}
	\end{questions}
	
\end{document}